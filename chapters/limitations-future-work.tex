%TEX root = ../dissertation.tex

\chapter{Limitations and Future Work}
\label{chapter:limitations-future-work}

During the development of this solution, and mainly due to compatibility issues, some limitations were raised. The most flagrant is the possibility of compromising a network node and sniffing the contents of the propagated packets. Although we are using \gls{LLSec} and this assures confidentiality, integrity, authenticity of the propagated packets while also providing authentication of the network nodes, in the event of compromising an already deployed node, the messages are cyphered and deciphered on every device, exposing the payload to the attackers. In the future a solution that protects the packets from the source all the way to the destination needs to be addressed so that the solutions can be totality secure against privacy attacks.\\
Additionally, and as discussed in Section \ref{sec:attack_analysis}, an attacked could try to introduce himself to the network by stealing the network credentials from a deployed device. The mitigations strategies for this attack can be either software based,  assuring that secure memory areas cannot be copied to external locations. Or hardware based, certain integrated circuits assure the stored information cannot be read from them \cite{Lesjak2014}. Although they have not been address in this system as we believe they are out of the scope of this project and more related to other computer science fields, they are still important and need to be address for increasing the security of the system globally.\\
Furthermore, the authors attempted to publish their work by writing two articles and sending them to two separate conferences. From the comments of the reviewers, additional limitations and improvement opportunities arose.\\
Regarding the article that was sent to the international conference (Appendix  \ref{appendix:acm_acsac}), the reviewers identified the bootstrapping process to be not adequate for massive deployment due to the fact that it needs to be physically connected to a central station and does not come shipped with the necessary credentials to start operating immediately. Although our goal was never a massive deployment, but scenarios with the dimension of a Smart Campus, this question loses some relevance. However it is still important to this about this system as possibly being used on a large scale enterprise solution and to that extent this issue needs to be addressed. In order to avoid the necessity of physical connections, one could attempt, in the future,to resort to the always required 802.15.4 radio as the way of sending the custom firmware to the new device. This would need to be done on a secure environment so that the firmware could not be sniffed by any attacker and from that firmware be able to steal the network credentials, but would reduce the time and hardware connections required for flashing a new device.\\
Regarding the article that was sent to the national conference (Appendix \ref{appendix:inforum_paper}), the reviewers were mostly concerned about scalability issues. Due to budget limitations, it was not possible to create a network larger than four nodes and so the capabilities of the system under heavy usage were not tested. This is important and needs to be studied before thinking about deploying this system on a large scale. To achieve that, one could try, in the future, to integrate this system with the IoT-LAB \footnote{https://www.iot-lab.info/} large scale infrastructure facility suitable for testing small wireless sensor devices and heterogeneous communicating objects. If such integration could be achieved, the system could be tested with dozens of nodes on real hardware objects.\\
The submitted article was accepted as an Exended Abstract (Appendix \ref{appendix:inforum_abstract}), and presented during both the conference track and the break hours as a poster (Appendix \ref{appendix:inforum_poster}).