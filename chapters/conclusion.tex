%!TEX root = ../dissertation.tex

\chapter{Conclusions and Future Work}
\label{chapter:conclusion}
Due to the limitations of \gls{IoT} devices, achieving secure communications is not an easy task. In order to allow the deployment of battery powered nodes, their communication model must be very efficient and consume the minimum amount of power required for operation. To achieve those requirements we started by analysing the existing protocols across the OSI layers, trying to find the best suited solutions for this type of environments. After a thorough comparison we achieved a working stack of protocols but soon discovered possible breaches and attacks, especially on the network layer. Those attacks were further investigated and catalogued. Given the common principle on the majority of the attacks, the introduction of rogue nodes to the network, we presented some possible solutions based on secure bootstrapping, the secure authentication of new nodes when joining a network.\\
Once the energy efficient stack, possible attacks and mitigation strategies were defined, we proposed our solution based on a Smart Campus scenario. This solution is focused on providing the joining devices all the secure credentials required for a secure bootstrapping before the deploy on the field, so that when they start the operation phase no additional credentials need to be fetched, implying that no additional energy is spent on configuration.\\
The achieved solution was then evaluated and the results confirm that it us suitable for operation since it fits the hardware commonly used in the \gls{IoT} while maintaining a low power consumption. Also, experiments show that the bootstrapping process can be done by users without inner knowledge of the system and without requiring much time to complete. Furthermore, although the network faces severe disruption in the presence of Wi-Fi networks, it can still operate under stress with diminished performance.\\
Although the achieved solution is working and fulfilling the proposed objectives there is room for improvement and additional work. As discussed in Section \ref{sec:attack_analysis}, an attacked could try to introduce himself to the network by stealing the network credentials from a deployed device. The mitigations strategies for this attack can be either software based,  assuring that secure memory areas cannot be copied to external locations. Or hardware based, certain integrated circuits assure the stored information cannot be read from them \cite{Lesjak2014}. Although they have not been address in this system as we believe they are out of the scope of this project and more related to other computer science fields, they are still important and need to be addressed for increasing the security of the system globally.\\


In order to avoid the necessity of physical connections, one could attempt, in the future,to resort to the always required 802.15.4 radio as the way of sending the custom firmware to the new device. This would need to be done on a secure environment so that the firmware could not be sniffed by any attacker and from that firmware be able to steal the network credentials, but would reduce the time and hardware connections required for flashing a new device.\\

Furthermore, the authors attempted to publish their work by writing two articles and sending them to two separate conferences. From the comments of the reviewers, additional limitations and improvement opportunities arose.\\
Regarding the article that was sent to the international conference (Appendix  \ref{appendix:acm_acsac}), the reviewers identified the bootstrapping process to be not adequate for massive deployment due to the fact that it needs to be physically connected to a central station and does not come shipped with the necessary credentials to start operating immediately. Although our goal was never a massive deployment, but scenarios with the dimension of a Smart Campus, this question loses some relevance. However it is still important to this about this system as possibly being used on a large scale enterprise solution and to that extent this issue needs to be addressed. 
Regarding the article that was sent to the national conference (Appendix \ref{appendix:inforum_paper}), the reviewers were mostly concerned about scalability issues. Due to budget limitations, it was not possible to create a network larger than four nodes and so the capabilities of the system under heavy usage were not tested. This is important and needs to be studied before thinking about deploying this system on a large scale. To achieve that, one could try, in the future, to integrate this system with the IoT-LAB \footnote{https://www.iot-lab.info/} large scale infrastructure facility suitable for testing small wireless sensor devices and heterogeneous communicating objects. If such integration could be achieved, the system could be tested with dozens of nodes on real hardware objects.\\
The submitted article was accepted as an Exended Abstract (Appendix \ref{appendix:inforum_abstract}), and presented during both the conference track and the break hours as a poster (Appendix \ref{appendix:inforum_poster}).

