%!TEX root = ../dissertation.tex

\chapter{Conclusion}
\label{chapter:conclusion}
Due to the limitations of \gls{IoT} devices, achieving secure communications is not an easy task. In order to allow the deployment of battery-powered nodes, their communication model must be very efficient and consume the minimum amount of power required for operation. To achieve those requirements we started by analysing the existing protocols across the OSI layers, trying to find the best suited solutions for this type of environments. After a thorough comparison we selected a working stack of protocols but soon discovered possible breaches and attacks, especially on the network layer. These attacks were further investigated and catalogued and we found that the common flaw on the majority the attacks was the introduction of rogue nodes to the network. We then presented some possible solutions based on secure bootstrapping, the secure authentication of new nodes when joining a network.\\
Once a working protocol stack was assembled, possible attacks and mitigation strategies were defined, and we proposed our solution based on a Smart Campus scenario. This solution is focused on providing the joining devices all the secure credentials required for a secure bootstrapping before the deploy on the field, so that when they start the operation phase no additional credentials need to be fetched, implying that no additional energy is spent on configuration.\\
The achieved solution was then evaluated and the results confirm that it is suitable for operation since it fits the hardware commonly used in the \gls{IoT} while maintaining a low power consumption. Also, experiments showed that the bootstrapping process can be done by users without inner knowledge of the system and without requiring much time to complete. Furthermore, although the network faces severe disruption in the presence of Wi-Fi networks, it can still operate under stress with diminished performance.

\section{Future Work}

Although the achieved solution is working and fulfilling the proposed objectives there is room for improvement and additional work. As discussed in Section \ref{sec:attack_analysis}, an attacker could try to introduce himself to the network by \emph{stealing the network credentials} from a deployed device. The mitigations strategies for this attack can be either software based,  assuring that secure memory areas cannot be copied to external locations, or hardware based, certain integrated circuits assure the stored information cannot be read from them \cite{Lesjak2014}. Although they have not been address in this system as we believe they are out of the scope of this project and more related to other computer science fields, they are still important and need to be addressed for increasing the security of the system globally.\par 
To further protect the network against this attack, future changes can also be performed on the management station. An improvement would be to \emph{keep track of the connected devices} and fire alerts when two devices with the same identifier started operating. Since there cannot be two devices with the exact same properties, this situation would mean that the firmware of an authorized node was somehow cloned into a rogue device.\par
Regarding the used \emph{hardware and interfacing} capabilities, one could attempt, in order to avoid the necessity of physical connections, to resort to the always required 802.15.4 radio as the way of sending the custom firmware to the new device. This would need to be done on a secure environment so that the firmware could not be sniffed by any attacker and from that firmware be able to steal the network credentials, but would reduce the time and hardware connections required for flashing a new device. Although our goal was never a massive deployment, but scenarios with the dimension of a Smart Campus, this question loses some relevance. However it is still important to this about this system as possibly being used on a large scale enterprise solution and to that extent this issue needs to be addressed.\par
Thinking about \emph{scalability}, due to the small amount of physical devices available for testing, we never constructed a network larger than four nodes, meaning the capabilities of the system under heavy usage were not tested. This is important and needs to be studied before thinking about deploying this system on a large scale. To achieve that, one could try, in the future, to integrate this system with the IoT-LAB \footnote{https://www.iot-lab.info/} large scale infrastructure facility suitable for testing small wireless sensor devices and heterogeneous communicating objects. If such integration could be achieved, the system could be tested with dozens of nodes on real hardware devices.


