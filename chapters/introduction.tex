%!TEX root = ../dissertation.tex

\chapter{Introduction}
\label{chapter:introduction}
The Internet of Things \gls{IoT} can be seen as a web of interconnected devices that go from everyday wearable objects into fully deployed sensor networks. Despite the huge variety and characteristics of these devices, one thing that they all have in common is the constrained nature that they are built upon. In order to enable the massive deployment to be expected in the near future,\footnote{http://blogs.wsj.com/cio/2015/06/02/internet-of-things-market-to-reach-1-7-trillion-by-2020-idc/} \gls{IoT} devices must be accessible and affordable, capable of operating under lossy wireless networks while being battery powered. This poses a challenge to current Internet protocols since the assumptions regarding the devices' capabilities and objectives do not hold true.\\ To allow the \gls{IoT} vision to come forward, several new protocols have been developed across the OSI layers, each addressing and tackling the challenges involved in trying to keep the quality and assurances of stronger, more expensive protocols, on constrained systems. After being thoroughly analysed, these protocols have been selected and grouped in a power-efficient stack, establishing a base line for power consumption.
Additionally, major attention has been given to information security because for both corporations and individuals, the interconnection of the devices around us can provide information about our choices and whereabouts, therefore leaking corporate information or simply reducing our individual privacy \cite{Ukil2015}. Thus, the focus moved towards adding mechanism to ensure authentication, confidentiality and integrity of the transmitted information by securing the communication channel. In order to understand the cost of adding these mechanisms, additional experiments have been performed so that the added power consumption can be measured, profiled and documented, enabling the finding of the best parameters and requirements for a desired level of security.\\

TODO: não tive inspiração suficiente, ainda, para escrever um paragrafo que diga que construimos uma solução assente num cenário smart campus porque é um bom modelo para testarmos o sistema e obtermos medições interessantes, e que também foi dado grande foco na facilidade de uso e monitorização do sistema, permiting a qualquer um sem conhecimentos técnicos fazer deploy da rede de sensores e monitorizar os valores reportados

\section{Document Roadmap}

In this document we start by analysing the state-of-the-art in Section \ref{sec:related_work}. This includes the selection of the most adequate protocol stack for our necessities in Section \ref{sec:protocol_analysis}, an overview of the existing attacks and mitigation strategies in Section \ref{sec:attack_analysis} and a summary of the existing solutions regarding secure insertion of new nodes in an existing network in Section \ref{sec:secure_bootstrapping}. All this knowledge will be integrated into our proposed solution defined in Section \ref{sec:proposed_solution}. Section \ref{sec:work_evaluation} defines how our work will be tested and evaluated so that a power-aware perspective can be achieved. Section \ref{sec:work_planning} states how the development of our solution will unwind over the next months and finally, Section \ref{sec:conclusion} presents the conclusion of this document.