%!TEX root = ../dissertation.tex

\chapter{Introduction}
\label{chapter:introduction}
The Internet of Things \gls{IoT} can be seen as a web of interconnected devices that go from everyday wearable objects into fully deployed sensor networks. Despite the huge variety and characteristics of these devices, one thing that they all have in common is the constrained nature that they are built upon. In order to enable the massive deployment to be expected in the near future,\footnote{http://blogs.wsj.com/cio/2015/06/02/internet-of-things-market-to-reach-1-7-trillion-by-2020-idc/} \gls{IoT} devices must be accessible and affordable, capable of operating under lossy wireless networks while being battery powered. Chapter \ref{sec:network_overview} presents an overview of the type of networks and scenarios under consideration.

	This poses a challenge to current Internet protocols since the assumptions regarding the devices' capabilities and objectives do not hold true. To allow the \gls{IoT} vision to come forward, several new protocols have been developed across the OSI layers, each addressing and tackling the challenges involved in trying to keep the quality and assurances of stronger, more expensive protocols, on constrained systems. After being thoroughly analysed in Section \ref{sec:protocol_analysis}, these protocols have been selected and grouped in a power-efficient stack, establishing a base line for power consumption. 
	
	Since \gls{IoT} environments can range from home to enterprise or even city environments, a breach in security could potentially leak important company activity or provide information about individuals' choices and whereabouts constituting a violation of privacy \cite{Ukil2015}. To this extent, a study on existing attacks for constrained devices was conducted on Section \ref{sec:attack_analysis} and a common mitigation strategy further examined in Section \ref{sec:secure_bootstrapping}. This strategy provided security assurances at the cost of an increased infrastructure complexity capable of managing the network devices and security credentials. This managing solution -- AutoStrap -- derived its name from the necessity of a simple, automated solution and is presented in Chapter \ref{chapter:proposed_solution} on a SmartCampus scenario. 
	
	In order to accurately define the type and amount of resources needed to create such a network, we deployed our protocol stack and management station of physical hardware and performed energy, space and time profiling as presented in Chapter \ref{chapter:evaluation}. Finally, Chapter \ref{chapter:conclusion} states our conclusions and opportunities for future work.