%!TEX root = ../dissertation.tex

\begin{otherlanguage}{portuguese}
\begin{abstract}
\abstractPortuguesePageNumber

A \gls{IdC} e a sua visão de ligar todos os dispositivos entre sí e à Internet apresenta-se como uma oportunidade para criar grandes redes de captura e partilha de informação. No entanto, instrusos podem debilitar e capturar estas redes aproveitando-se dos seus limitados recursos por lançar uma larga gama de ataques aos seus elementos. No nosso trabalho abordamos este problema numa perspectiva energética, com o objectivo de encontrar a melhor relação entre segurança e consumo energético. Para atingir este objectivo, realizámos uma análise extensiva aos protocolos, ataques e estratégias de mitigação existentes, combinando essa informação no nosso sistema de gestão e arranque seguro de redes -- AutoStrap -- avaliado num cenário SmartCampus. Ademais, conduzimos medições ao nível do espaço utilizado e recursos consumidos para entender quais os recursos fisicos necessários para este tipo de aplicações.

% Keywords
\begin{flushleft}

\palavrasChave{Internet das Coisas, Segurança Energéticamente Ponderada, Aranque Seguro de Redes, CoAP, MQTT, 6LoWPAN, RPL, IEEE 802.15.4}

\end{flushleft}

\end{abstract}
\end{otherlanguage}
