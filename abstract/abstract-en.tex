%!TEX root = ../dissertation.tex

\begin{otherlanguage}{english}
\begin{abstract}
% Set the page style to show the page number
\thispagestyle{plain}
\abstractEnglishPageNumber


The \gls{IoT} and its vision of connecting every device to one another and to the Internet presents an opportunity to create large information sharing networks. However, intruders can capture and take advantage of the \gls{IoT} devices' constrained nature to disrupt these networks and launch a wide range of attacks on its nodes. In our work we address this issue from a power-aware perspective, identifying best relation between security and resource consumption. To achieve this objective we do a thorough analysis of the existing protocols, attacks and mitigation strategies, combining that information into our proposed bootstrapping and network management system -- AutoStrap -- evaluated on a Smart Campus scenario. Furthermore, we perform energy, space and time profiling of the network devices to endow future users with the knowledge of what kind of physical resources to deploy based on the proposed application security level.

% Keywords
\begin{flushleft}

\keywords{Internet of Things, Power-Aware Security, Secure Bootstrapping, CoAP, MQTT, 6LoWPAN, RPL, IEEE 802.15.4}

\end{flushleft}

\end{abstract}
\end{otherlanguage}
